% !TeX encoding = UTF-8
%% \textbf{重庆大学}通用毕业论文\LaTeXe{}模板
%%% 使用前请先阅读使用文档和用户协议,内有详细介绍。Happy Texing! :)
%% =======================================================
\documentclass%
	[type=bachelor, bilinguallist=apart, 
	printmode=auto, openany]{cquthesis}%
% 可用选项:
% type=[bachelor|master|doctor],      % 必选,毕业论文类型,以下项目不填时为默认
% liberalformat,                      % 可选,仅适用本科生,使用文学类论文标题格式,默认未打开
% proffesionalmaster=[true|false],    % 可选,仅适用研究生,是(true)否(false)专业硕士,默认为否
% printmode=[oneside|twoside|auto],	  % 可选,论文打印方式,默认采用auto按页数要求自动判定
% openany,|openright,                 % 可选,双面打印时每章的第一页仅右页开启,默认右页开启(openright)
% bilinguallist=[off|combined|apart], % 可选,图录表录等分别按双语题注混编(combined),分开编录(apart),默认关(off)
% blindtrail,                         % 可选,盲审模式,开启后封面姓名和致谢部分会隐藏,详情请参阅用户文档,默认关
% draft,                              % 写作期间可选,不渲染图片,关闭外围功能,加快预览速度,默认未开启

% 请在cquthesis.sty文件中定义其他会用到的宏包和自己的变量
% 这样可以防止main.tex太过臃肿。
\usepackage{cquthesis}

% 定义所有的图片文件在 figures 子目录下
\graphicspath{{figures/}}

% 定义数字圆
\usepackage{tikz}
\newcommand*\circled[1]{\tikz[baseline=(char.base)]{
            \node[shape=circle,draw,inner sep=1pt] (char) {#1};}}

%*** 写作时,使用这个命令只渲染你想查看的部分,提升工作效率,定稿时注释掉整行
%\includeonly{contents/experiment,contents/analysis,}


\begin{document}

\input{contents/cover}

\frontmatter %%%前置部分(封面后绪论前)
%\cquauthpage[contents/cover1.pdf]
%\cquauthpage[contents/cover2.pdf]
%\cquauthpage[contents/cover3.pdf]
%\cquauthpage[contents/cover4.pdf]

%% 原创声明和授权说明书,可选:用扫描页替换
%\cquauthpage[authscan.pdf]
%\cquauthpage

%% 摘要
\makeabstract

%% 目录,注意需要多次编译才能更新
\setlength{\cftbeforetoctitleskip}{0pt}
\setlength{\cftaftertoctitleskip}{20pt}
\tableofcontents

\setlength{\cftbeforelottitleskip}{0em}
%% 插图索引,可选,如不用可注释掉
%\renewcommand*{\listfigurename}{插图索引}
% \clearpage
% \phantomsection  
% \addcontentsline{toc}{chapter}{插图索引}
% \listoffigures
%\listoffiguresEN
%% 表格索引,可选
%\renewcommand*{\listtablename}{表格索引}
% \clearpage
% \phantomsection  
% \addcontentsline{toc}{chapter}{表格索引}
% \listoftables
%\listoftablesEN
%% 公式索引,可选
%\listofequations
%\listofequationsEN
%% 符号对照表,可选
% \clearpage
% \phantomsection 
% \addcontentsline{toc}{chapter}{主要符号对照表}
% \input{contents/denotation}
% %% 缩略语对照表,可选
% \clearpage
% \phantomsection 
% \addcontentsline{toc}{chapter}{缩略语对照表}
% \input{contents/abbreviate}

\mainmatter %%% 主体部分(绪论开始,结论为止)
%* 子文件的多少和内容由你决定(最好以章为单位),基本原则是提速预览、脉络清晰、管理容易。

% 设置字号为小四
\renewcommand{\normalsize}{\fontsize{12pt}{20pt}\selectfont}
% 设置小四正文行间距为 20 磅
\setstretch{1.312}

\include{contents/introduction}
\chapter[\hspace{0pt}基础知识]{{\CJKfontspec{SimHei}\zihao{3}\hspace{-5pt}基础知识}}\label{section 2}


基础知识
\chapter[\hspace{0pt}算法]{{\CJKfontspec{SimHei}\zihao{3}\hspace{-5pt}算法}}\label{section 2}


测试
\chapter[\hspace{0pt}分类应用]{{\CJKfontspec{SimHei}\zihao{3}\hspace{-5pt}分类应用}}\label{section 2}


UCI数据集分类应用
\chapter[\hspace{0pt}医疗应用]{{\CJKfontspec{SimHei}\zihao{3}\hspace{-5pt}医疗应用}}\label{section 2}


医疗心跳序列上的应用
% \include{contents/architecture}
% \include{contents/resource}
% \include{contents/qos}
% \include{contents/system}
\include{contents/conclusion}
%\include{contents/yourFreeChoise}


\backmatter %%% 后置部分(致谢、参考文献、附录等)

%% 参考文献
% 顺序编码制:cqunumerical		
% 注意:至少需要引用一篇参考文献,否则下面两行会引起编译错误。
%\bibliographystyle{cqunumerical}
\bibliographystyle{gbt7714-numerical}
\bibliography{ref/refs}


% %% 附录(按ABC...分节,证明、推导、程序、个人简历等)
% \appendix
% \include{contents/appendix}


%% 致谢
\include{contents/ack}

\end{document}